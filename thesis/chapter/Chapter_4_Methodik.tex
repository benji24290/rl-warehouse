\chapter{Methodik}
Zur Beantwortung der zwei Forschungsfragen werden zwei Experimente aufgebaut. Dabei wird der Design-Science-Ansatz verfolgt, welcher von Hevner et al. in \cite{dsr} beschrieben wird. Folgende Guideline wird dabei verwendet:\\
\textbf{Design as an Artifact:}\\
Das Environment ist dabei das Artefakt, mithilfe dessen die zwei Forschungsfragen beantwortet werden. \\
\textbf{Problem-Relevance:}\\
Die Relevanz des Problems wurde in der Einleitung bereits aufgezeigt und besteht darin, dass eine mögliche Optimierung von Lagerverwaltungen durch maschinelles Lernen Kosten und Zeitersparnisse erzielen könnte.\\
\textbf{Design-Evaluation:}\\
Zum Evaluieren wird eine experimentelle Methode angewendet; es wird für jede Forschungsfrage ein kontrolliertes Experiment erstellt, in welchem die Forschungsfrage überprüft wird.\\
\textbf{Research-Contribution:}\\
Der wissenschaftliche Beitrag besteht neben den beantworteten Forschungsfragen, welche einen ersten Eindruck über die Anwendbarkeit von Reinforcement Learning in Lagerverwaltungen bieten sollen, auch aus dem Environment, welches sich als Artefakt identifiziert. Das Environment soll als erweiterbare Grundlage dienen, um weiterführende Forschung in diesem Bereich durchzuführen.\\
\textbf{Research-Rigor}\\
In der Konstruktions- sowie in der Evaluationsphase müssen Methoden angewendet werden, welche sich bereits etabliert haben. Dabei hat sich der Verfasser beim Erstellen des Environments an die Vorgaben von OpenAI \cite{gym} gehalten. Zum Evaluieren werden die zwei grundlegenden Algorithmen für Reinforcement Learning verwendet.\\
\textbf{Design as a Search-Process:}\\
Die Iterationen, welche in dieser Arbeit erfasst werden, behandeln eine simplifizierte Version des Problems und dessen Überprüfung.
\newpage
\noindent\textbf{Communication of Research:}\\
Die Erkenntnisse müssen dem technischen Publikum in Form eines reproduzierbaren Artefakts übermittelt werden, welches anschliessend für die Anwendung oder – in diesem Fall – für weiterführende Forschungen zur Verfügung steht. Ausserdem wird dem Publikum aus dem Management-Bereich das Wissen vermittelt, inwiefern das Artefakt in ihrem Problem angewendet werden kann.\\
\smallskip\\
\textbf{Reproduzierbarkeit}\\
Sämtliche Experimente wurden in einem separaten Branch im Repository festgehalten, was die Reproduzierbarkeit der Experimente drastisch verbessern soll. Der entsprechende Link ist jeweils am Ende jedes Experimentes zu finden. Im README.md sind die Abhängigkeiten sowie die Anweisungen zum Reproduzieren des Experiments dokumentiert.
